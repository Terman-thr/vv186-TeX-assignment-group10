\documentclass[11pt,twoside,a4paper]{article}
\usepackage{amsmath}
\usepackage{amssymb}
\usepackage{geometry}
\usepackage{multicol}% multi colomn
\usepackage{graphicx}
\usepackage{float}
\usepackage{cases}
\geometry{a4paper,left=2cm,right=2cm,top=1cm,bottom=2cm}
\DeclareMathSizes{12}{20}{14}{10}
\begin{document}
\title{Ex9 186 Fall}
\author{Assignment group 10}
\date{}
\maketitle
\section{Exercise 9.1}
\subsection{i}
First, we use the l'Hopital's rule:
$$ \lim_{x \to \infty} \frac{e^x-e^{-x}}{e^x+e^{-x}}
    =\lim_{x \to \infty} \frac{e^x+e^{-x}}{e^x-e^{-x}} $$
but we cannot prove that
$\displaystyle \lim_{x \to \infty} \frac{e^x+e^{-x}}{e^x-e^{-x}}$
and $\displaystyle \lim_{x \to \infty} \frac{e^x-e^{-x}}{e^x+e^{-x}}$
necessaily exist by just using l'Hopital's rule.
\newline
Another approach:
\par
Because $e^{-x}>0 $ is always true. So we have
$$\lim_{x \to \infty} \frac{e^x-e^{-x}}{e^x+e^{-x}}
    =\lim_{x \to \infty} \frac{e^{2x}-1}{e^{2x}+1}=1 $$

\subsection{ii}
First, we use the l'Hopital's rule:
$$ \lim_{x \to 0} \frac{x^2\cos (\frac{1}{x})}{\sin x}
    =\lim_{x \to 0} \frac{2x\cos (\frac{1}{x})+\sin (\frac{1}{x}) }{\cos x} $$
However, denominator $\cos x$ does not converge to $0$ or $+\infty$.
So, l'Hopital's rule cannot be used.
\newline
Another approach:
\par
$$ \lim_{x \to 0} \frac{x^2\cos (\frac{1}{x})}{\sin x}
    =\lim_{x \to 0} (\frac{x}{\sin x})\cdot \lim_{x \to 0} (x\cos(\frac{1}{x}))=1\times 0=0$$

\subsection{iii}
\begin{equation}
    \begin{aligned}
        \lim_{x \to \infty} \frac{f'(x)}{g'(x)}
         & = \lim_{x \to \infty} \frac{1+\cos ^2x-\sin ^2 x}{ e^{\sin x}\cos xf(x)+f'(x)e^{\sin x}} \\
         & =\lim_{x \to \infty}  \frac{2\cos x}{ e^{\sin x}(x+\sin x\cos x +2\cos x)}
    \end{aligned}
\end{equation}
Because $\displaystyle \lim_{x \to \infty} e^{\sin x}(x+\sin x\cos x +2\cos x)
    =\lim_{x \to \infty} (e^{\sin x}x+e^{\sin x} (\sin x\cos x +2\cos x))$,
and $e^{\sin x} \ge e^{-1}$. Thus,
$\displaystyle \lim_{x \to \infty} (e^{\sin x}x+e^{\sin x} (\sin x\cos x +2\cos x))=\infty$.
\newline
So $\displaystyle \lim_{x \to \infty} \frac{f'(x)}{g'(x)}=0$
\newline
Firstly,
$$\lim_{x \to \infty} \frac{f(x)}{g(x)}
    =\lim_{x \to \infty} \frac{f(x)}{f(x)e^{\sin x}}  $$
Because when $x\rightarrow \infty, f(x)\neq 0$,
So,
$\displaystyle \lim_{x \to \infty} \frac{f(x)}{g(x)}
    = \lim_{x \to \infty} e^{-\sin x} $.
Because $\displaystyle \lim_{x \to \infty} e^{-\sin x}$ does not exist.
Thus, $\displaystyle \lim_{x \to \infty} \frac{f(x)}{g(x)}$ does not exist.
And this doesn't contradict with l'Hopital's rule.

\section{Exercise 9.2}
\subsection{i}
The function we find is $$f(x)=\sin (x^2) \quad x\in \mathbb{R} $$
We can know that $\displaystyle\sup_{x\in \mathbb{R}} |f(x)|=|\sin (x^2)|=1$,
and $$\sup_{x\in \mathbb{R}} |f'(x)|=\sup_{x\in \mathbb{R}} |x^2\cdot\cos (x^2)|
    =\infty \quad \text{as } x \rightarrow \infty$$

\subsection{ii}
The function we find is
$$f(x)=\ln (x) \quad x\in \mathbb{R} $$
We can know that $\displaystyle\sup_{x\in \mathbb{R}} |f(x)|=|\ln (x)|=\infty$, and
$$\sup_{x\in \mathbb{R}} |f'(x)|=\sup_{x\in \mathbb{R}} |\frac{1}{x}|
    =0 \quad \text{as } x \rightarrow \infty$$




\section{Exercise 9.5}
\subsection{9.5.1}
Because $z^7=3+4i$, so $|z|^7=5$ and we can know $|z|=\sqrt[7]{5}:=a $
Then, we assume that\newline
$z=a\cos \theta +a i \sin \theta $ $\theta $ is a fixed number that lays in
the interval of $[0,2\pi)$. So we can have the equation :
$$7\theta =\arcsin (\frac{4}{5})+2k\pi \quad k\in \mathbb{N}$$
$$\theta =\frac{\arcsin (\frac{4}{5})+2k\pi}{7} \quad k\in \mathbb{N}$$
We can know there are $k=0,1,2,3,4,5,6$ that can make the equation
correct.

\subsection{9.5.2}
Assume that $z=:a+bi\quad a,b\in \mathbb{R}$, we can substitute it into
the equation and we get 
$$(a^2-b^2+b+1)+(2ab-a)i=0$$
Then, we know :
$$
\begin{cases}
    a^2-b^2+b+1=0\\
    2ab-a=0
\end{cases}
$$
1.When $a=0$, we can also know $b^2-b-1$ and thus $\displaystyle
b=\frac{1\pm \sqrt{5}}{2}$.
\newline
2.When $a\neq 0$, we can also know $\displaystyle b=\frac{1}{2}$, we can thus get
 $\displaystyle a^2+\frac{5}{4}=0$, which contradicts with $a \in \mathbb{R}$
\newline
$$z=\frac{1+\sqrt{5}}{2} i\quad \text{or} \quad z=\frac{1-\sqrt{5}}{2}i$$

\subsection{9.5.3}
We assume $z^2=:x+yi\quad (x,y\in \mathbb{C})$ and $z=:a+bi\quad (a,b\in \mathbb{R})$
\begin{equation}
    \begin{aligned}
        (z^2)^2+z^2+1&=0\\
        z^2=-\frac{1}{2}\pm \frac{\sqrt{3}}{2}i
    \end{aligned}
\end{equation}
$z^2=(a^2-b^2)+2abi$, so we can know:
$$
\begin{cases}
    a^2-b^2=-\frac{1}{2}\\
    ab=\pm \frac{\sqrt{3}}{4}
\end{cases}
$$
We can know $\displaystyle a=\pm\frac{\sqrt{3}}{4b}$ and substitute it into the 
equation set. We can know 
$
\begin{cases}
    \displaystyle
    a=\frac{1}{2}\\
    \displaystyle
    b=\frac{\sqrt{3}}{2}
\end{cases}
$
and 
$
\begin{cases}
    \displaystyle
    a=-\frac{1}{2}\\
    \displaystyle
    b=-\frac{\sqrt{3}}{2}
\end{cases}
$

\subsection{9.5.4}
$$
\begin{cases}
    iz-(1+i)w=3\\
    (2+i)z+iw=4
\end{cases}
$$
We can get:
$$
\begin{cases}
    -z-i(1+i)w=3i\\
    (1+i)(2+i)z+i(1+i)w=4
\end{cases}
$$
We can solve the equation set and get the solution:
$
\begin{cases}
    \displaystyle
    z=\frac{7}{3}-\frac{4}{3}i\\
    \displaystyle
    w=\frac{1}{3}+2i
\end{cases}
$

\section{9.6}
\subsection{9.6.1}
\begin{equation}
    \begin{aligned}
    \tan (x+y)&=\frac{\sin (x+y)}{\cos (x+y)} \\
        &=  \frac{\sin x \cos y+\sin y \cos x}{\cos x \cos y-\sin x\sin y} 
        \quad\text{(divide numerator and denominator by $\cos x \cos y$)}\\
        &=\frac{\tan x+\tan y}{1-\tan x\tan y}
    \end{aligned}
\end{equation}
Because $\cos (x+y) \neq 0$ and $\tan (x+y)$ exists. Thus, we can get 
$\displaystyle x+y\neq \frac{\pi}{2}+k\pi\quad (k\in \mathbb{Z})$

\subsection{9.6.2}
$$\arctan x+\arctan y=\arctan (\frac{x+y}{1-xy})$$
Assume $\displaystyle x:=\tan \alpha ,y:=\tan \beta 
\quad (\alpha ,\beta \in (-\frac{\pi}{2},\frac{\pi}{2}),x,y\in \mathbb{R},x\cdot y \neq1)$, 
Thus, we can get 
$$
\arctan (\tan \alpha )+\arctan (\tan \beta )
-\arctan (\frac{\tan \alpha +\tan \beta }{1- \tan \alpha \tan \beta })
=\alpha +\beta -\arctan (\tan (\alpha +\beta ))=0
$$
So, we get the proof.

\subsection{9.6.3}
We first establish that:
$$
\frac{d}{dx}(\arctan x+ \arctan \frac{1}{x})=\frac{1}{x^2+1}-\frac{1}{x^2}\cdot\frac{1}{1+(\frac{1}{x})^2}=0
$$
Thus, $\forall x \in \mathbb{R}$, for some $c \in \mathbb{R}$
$$
\arctan x +\arctan\frac{1}{x}=c
$$
Then, $\forall x\in \mathbb{R}$, 
since $\displaystyle \arctan 1=\frac{\pi}{4}$, we see that
$$
\arctan x+ \arctan \frac{1}{x}=\arctan 1+ \arctan 1=\frac{\pi}{4}+\frac{\pi}{4}=\frac{\pi}{2}
$$




\end{document}