\documentclass[11pt,twoside,a4paper]{article}
\usepackage{amsmath}
\usepackage{amssymb}
\usepackage{geometry}
\usepackage{multicol}% multi colomn
\geometry{a4paper,left=2cm,right=2cm,top=1cm,bottom=2cm}
\DeclareMathSizes{12}{20}{14}{10}


%Ex2.1
\begin{document}
\title{Ex2 186 Fall}
\author{Assignment group 10}
\date{}
\maketitle
\section{Exercise 2.1}
\subsection{i}
We can learn from the quetion that $\displaystyle \frac{m^2}{n^2}<2$ and $m,n\in\mathbb{N}^*$.Thus:
$$m^2<2n^2$$
Then we can add same algebraic epression on both side to make it still right.
$$(m^2+2n^2+4mn)+m^2<(m^2+2n^2+4mn)+2n^2$$
then we can get :    
$2m^2+4mn+2n^2<m^2+4mn+4n^2$
$$2(m+n)^2<(m+2n)^2$$
because both sides are positive, and we gain the proof.
\[\frac{(m+2n)^2}{(m+n)^2}>2\]\par
For the latter  inequation, first we can get $m+n>n$,because both sides are positive interger, we can get:
$(m+n)^2>n^2$ \par
Similarly, 
$$\frac{1}{(m+n)^2}<\frac{1}{n^2}$$
because $m^2<2n^2$, so $-m^2+2n^2>0$, thus we can multiply both sides by $-m^2+2n^2$ 
$$\frac{-m^2+2n^2}{(m+n)^2}<\frac{-m^2+2n^2}{n^2}$$
because $\displaystyle\frac{-m^2+2n^2}{(m+n)^2}=\frac{(m+2n)^2}{(m+n)^2}$ and $\displaystyle\frac{-m^2+2n^2}{n^2}=2-\frac{m^2}{n^2}$\par
Thus,we gain the proof:$$\displaystyle\frac{(m+2n)^2}{(m+n)^2}-2<2-\frac{m^2}{n^2}$$



%Ex2.1
\subsection{ii}
We can learn from the quetion that $\displaystyle\frac{m^2}{n^2}>2$ and m,n$\in\mathbb{N}^*$.Thus:
$$m^2>2n^2$$
Then we can add same algebraic epression on both side to make it still right.
$$(m^2+2n^2+4mn)+m^2>(m^2+2n^2+4mn)+2n^2$$
then we can get :    
$2m^2+4mn+2n^2>m^2+4mn+4n^2$
$$2(m+n)^2>(m+2n)^2$$
because both sides are positive, and we gain the proof.
\[\frac{(m+2n)^2}{(m+n)^2}<2\]
For the latter  inequation, first we can get $m+n>n$,because both sides are positive interger, we can get:
$(m+n)^2>n^2$ \par
Similarly, 
$$\frac{1}{(m+n)^2}<\frac{1}{n^2}$$
because $m^2<2n^2$, so $-m^2+2n^2<0$, thus we can multiply both sides by $-m^2+2n^2$ and change the inequilty sign to its opposite.
$$\frac{-m^2+2n^2}{(m+n)^2}>\frac{-m^2+2n^2}{n^2}$$
because $\displaystyle\frac{-m^2+2n^2}{(m+n)^2}=\frac{(m+2n)^2}{(m+n)^2}$ and $\displaystyle\frac{-m^2+2n^2}{n^2}=2-\frac{m^2}{n^2}$\par
Thus,we gain the proof:$$\displaystyle\frac{(m+2n)^2}{(m+n)^2}-2>2-\frac{m^2}{n^2}$$


\subsection{iii}
Let  $\displaystyle s=\frac{m}{n}$,  because  $m,n \in \mathbb{N}^* $
So we can know  $s>0\,(s \in \mathrm{Q}) ,$\par\noindent
If $0<s\le1$,obviously,there exists $\displaystyle (\frac{m'}{n'})^2<2$.\par\noindent
If $s>1,s^2<2$,let $\displaystyle A = s + \frac{2-s^2}{3s}$,so  $A>s,A \in\mathrm{Q}$
Then, we can get:
\begin{align*}
 A^2&=s^2+(\frac{2-s^2}{3s})^2+\frac{4-2s^2}{3} 
\\&=s^2+\frac{(2-s^2)^2}{9s^2}+\frac{4-2s^2}{3}
\end{align*}
Because $3s^2>2-s^2,$ thus$\displaystyle (\frac{2-s^2}{3s})^2<\frac{2-s^2}{3}.$\par\noindent
Thus,we can know:
$$A^2<s^{2}+\frac{2-s^{2}}{3}+\frac{4-2 s^{2}}{3}=2 $$
So ,let $\displaystyle A=\frac{m'}{n'}$,and we can know A satisfy the requirement.





\subsection{iv}
Let$\,\,y=\min U_{2}\,\,\,(y \in  \mathrm{Q}) $, so we can get 
$\displaystyle y^{2}>2, y>\frac{2}{y} $.
\par\noindent  Then,$$(y-\frac{2}{y})^{2}>0$$ $$ y^{2}+\frac{4}{y^{2}}>4 $$
Then,let 
$\displaystyle s=\frac{1}{2}(y+\frac{2}{y}) \quad (s \in ) $\par\noindent
So we can get:$$ s^{2}=\frac{1}{4}(y^{2}+\frac{4}{y^{2}}+4) >2 $$ Thus$\,\,s \in  U_{2}$.\par\noindent
However,
$$s=\frac{1}{2}(y+\frac{2}{y})<\frac{1}{2}(y+y) =y=minU_{2}$$
So we can know $minU{_2}\,\,$doesn't exist in $\mathbb{Q}$.


\subsection{v}
Assume there exists $y \in \mathbb{Q}$,and $y=infU{_2}$. \par\noindent Firstly, if $y^2>2$, according to Ex2.1 iv) ,there always exists $\displaystyle s=\frac{1}{2}(y+\frac{2}{y})<y\,\,\,\,(s\in  \mathbb{Q})$ to make $y^2>s^2>2$ correct, which leads to a contradiction.
\par\noindent
Secondly, if $y^2=2$ assume $\displaystyle y = \frac{p}{q}\quad (p,q\in  \mathbb{N}^*)$(p and q are relatively prime to each other), so $\displaystyle y^2=\frac{p^2}{q^2}=2$, so we can get $p^2=2q^2$. Then we know p is an even number.Thus, let  $p=2k(k\in  \mathbb{N})$. So, $4k^2=2q^2$, which is  $2k^2=q^2$. Thus ,we can know q is also a even number, which contradicts to our assumption that p and q are relatively prime to each other(with the commerdivisor of 2).
\par\noindent
Thirdly, if $y^2<2$,according to Ex2.1 iii) ,there always exists $\displaystyle s=\frac{1}{2}(y+\frac{2-y^2}{3y})>y\,\,\,\,(s\in  \mathbb{Q})$ to make $y^2<s^2<2$ correct, which leads to a contradiction.




%Ex2.2
\section{Exercise 2.2}
\begin{multicols}{2}
$i)\,\,\,\,\,$maximun:$\displaystyle\frac{3}{2}$\par  minimun:N/A \par supremum:$\displaystyle\frac{3}{2}$ \par infimum:1
\end{multicols}
\begin{multicols}{2}
$ii)\,\,\,\,\,$maximun:$\displaystyle\frac{5}{4}$\par  minimun:N/A \par supremum:$\displaystyle\frac{5}{4}$ \par infimum:$-1$
\end{multicols}



%Ex2.3
\section{Exercise 2.3}
We have alreadly learnt that $y= inf\{t\in \mathbb{R}:t>0\land t^2>x\}$.  From this we can know $x>0, y^{2}<x$. 
$\displaystyle \text { Let } A=y+\frac{\Delta}{n y}\quad (\Delta=x-y^{2} ,n \in  \mathbb{N}^*)\\$
\begin{align*}
\left(y+\frac{\Delta}{n y}\right)^{2} &=\left(y+\frac{\Delta}{n y}\right)^{2} \\
&=y^{2}+\frac{\Delta^{2}}{n^{2} y^{2}}+\frac{2 \Delta n y^{2}}{n^{2} y^{2}} \\
&=y^{2}+\frac{\left(2 n y^{2}+\Delta\right) \Delta}{n^{2} y^{2}}
\end{align*}
Apparently,$\displaystyle\frac{(2 n y^{2}+\Delta) }{n^{2} y^{2}}>0$.Then, we will show that there $\exists n \in \mathbb{N}^*$,makes$\displaystyle\frac{(2 n y^{2}+\Delta) }{n^{2} y^{2}}<1$.\par\noindent
If $ \displaystyle 0<\frac{2 n y^{2}+\Delta}{n^{2} y^{2}}<1$,  because  $y^{2}>0 $,
let  $\displaystyle t=\frac{\Delta}{y^{2}}(t\in \mathbb{Q},t>0)$,and then we can  get
 $\displaystyle\frac{2 n+t}{n^{2}}<1 $.
\par\noindent
So, we can get:
$$n^{2}-2 n-t>0$$
Let $n=[t]+10$\,\,($[t]$ means the largest interger that smaller than $t$)
\par\noindent
Then we can get the left side:
\begin{align*}
left&=([t]+10)^2-2[t]-20-t  \\ &=[t]^2+18[t]-t+80
\end{align*}
If $t\in(0,1)$  ,  $left=80-t>0$
\par\noindent
If $t\in[0,\infty)$, $left=[t]^2+18[t]-t+80>[t]^2+16[t]+80>0$\par\noindent
So we prove that there exists $n=[t]+10$ to make $n^{2}-2 n-t>0$ correcct.\par\noindent
And thus, we get the proof that
$\exists n \in \mathbb{N}^*$,makes$\displaystyle\frac{(2 n y^{2}+\Delta) }{n^{2} y^{2}}<1$.  So when $n=[t]+10$, $$A^2=(y+\frac{\Delta}{n y})^{2}=y^{2}+\frac{(2 n y^{2}+\Delta) \Delta}{n^{2} y^{2}}<y^2+\Delta=x$$
Thus, we find it contradict with $y= inf\{t\in \mathbb{R}:t>0\land t^2>x\}$ and we gain the proof.





%Ex2.4
\section{Exercise 2.4}
\subsection{i}
$a)\,\,\,\,\,$almost upper bound:$(1,+\infty)$\quad  almost lower bound:$(-\infty,1]$ \par\noindent
$b)\,\,\,\,\,$almost upper bound:$(1,+\infty)$\quad  almost lower bound:$(-\infty,-1]$ \par\noindent
$c)\,\,\,\,\,$almost upper bound:$(0,+\infty)$\quad  almost lower bound:$(-\infty,0)$ \par\noindent
$d)\,\,\,\,\,$almost upper bound:$[\sqrt{2},+\infty)$\quad  almost lower bound:$(-\infty,0]$ \par\noindent

\subsection{ii}
Because X is a bounded infinite set, so there must exist $x = supX$, so there exists one or doesn't exist $y$ that $y \ge x$. Thus, x is a almost upper bounds of $X$, and $Y$ is nonempty.
\par\noindent
Then let $a=infX$,  if $\exists y \in Y$ such that $y\ge a=infX$, because $|X|=+\infty$, there will be infinite elements that larger than y, which contradict to  "there are only finitely many numbers $y \in A$ with $y \ge x$"in the question. So, $infY>infX$, and we can get Y is bounded below.


\subsection{iii}
$\overline{lim}\{1+2^{-n}: n \in \mathbb{N}^{*}   \}=1$ \par\noindent
$\displaystyle \overline{lim}\{(-1)^n +\frac{1}{n^2}:n \in  \mathbb{N}^{*}  \}=1$ \par\noindent
$\displaystyle \overline{lim}\{\frac{1}{n}: n \in \mathbb{Z}  \backslash \{0\}   \}=0$ \par\noindent
$\displaystyle \overline{lim}\{x \in \mathbb{Q}:0 \le x \le \sqrt{2}   \}=\sqrt{2}$ \par\noindent

\subsection{iv}
$Definition$ : By (P13), the superior  supY exists; this number is called the limit infimum of X and denoted by $lim supX$ or $\underline{lim}X$. 
$\underline{lim}\{1+2^{-n}: n \in \mathbb{N}^{*}   \}=1$ \par\noindent
$\displaystyle \underline{lim}\{(-1)^n +\frac{1}{n^2}:n \in  \mathbb{N}^{*}  \}=-1$ \par\noindent
$\displaystyle \underline{lim}\{\frac{1}{n}: n \in \mathbb{Z}  \backslash \{0\}   \}=0$ \par\noindent
$\displaystyle \underline{lim}\{x \in \mathbb{Q}:0 \le x \le \sqrt{2}   \}=0$ \par\noindent


\subsection{v}
\subsubsection{a)}
We assume that $\exists y =\underline{lim}(A) $ and $\exists z =\overline{lim}(A) $,which makes $y>z$.According to the difinition: in the interval $[inf(A),y]$, there only exists finite numbers. Similiarly, in the interval $[z,sup(A)]$, there also only exists finite numbers. Because $y>z$, so $[inf(A),y] \cup [z,sup(A)]=A$ ,and we can know the union set of these two finite set is also a finite set, which contradict with the question that A is a infinite set. Thus the assumption is wrong. Thus, we get the proof $\underline{lim}(A) \le \overline{lim}(A) $.

\subsubsection{b)}
Assume set $Y$ is the set of all almost upper bounds of $A$. Let $U{_1}$ be the set of the collection of the upper bounds of $A$, and let $U{_2}=Y \backslash  U{_1} $. So, we can get $Y =U{_1} \cup U{_2} $. Thus, we can know that all the elements in $ U{_2}$ are smaller than $supA$ and those in $ U{_1}$.\par\noindent
If $ U{_2} \neq \varnothing$, then we can get: $$inf Y=infU{_2}<infU{_1}<sup(A)$$
If $ U{_2} = \varnothing$, then we can get: $$inf Y=infU{_1}=sup(A)$$
Thus, we get the proof : $\overline{lim}(A)\le sup(A)$\par\noindent

Similiarly, assume set $Y$ is the set of all almost lower bounds of $A$. Let $U{_1}$ be the set of the collection of the lower bounds of $A$, and let $U{_2}=Y \backslash  U{_1} $. So, we can get $ Y =U{_1} \cup U{_2} $. Thus, we can know that all the elements in $ U{_2}$ are larger than $supA$ and those in $ U{_1}$.\par\noindent
If $ U{_2} \neq \varnothing$, then we can get: $$sup Y=supU{_2}>supU{_1}=inf(A)$$
If $ U{_2} = \varnothing$, then we can get: $$supY=supU{_1}=inf(A)$$
Thus, we get the proof : $\underline{lim}(A)\ge inf(A)$\par\noindent

\subsubsection{c)}
If $\overline{lim}A < supA$, there must $\exists b \in A$, which makes $supA>b>\overline{lim}A$  (or $supA$ should be smaller and there will be $supA=\overline{lim}A$ ,which contradict to the question).
According to the definition of $\overline{lim}A$, there only exists finite numbers(let it be $n$) that in the interval of $(\overline{lim}A,supA)$. Let these numbers be $b{_1},b{_2}...b{_n}$. Thus, $maxA=max\{b{_1},b{_2}...b{_n}\}$.So, we gain the proof.
\par\noindent
If $\underline{lim}A > supA$, there must $\exists b \in A$, which makes $infA<b<\underline{lim}A$  (or $infA$ should be larger and there will be $infA=\underline{lim}A$ ,which contradict to the question).
According to the definition of $\underline{lim}A$, there only exists finite numbers(let it be $n$) that in the interval of $(infA,\underline{lim}A)$. Let these numbers be $b{_1},b{_2}...b{_n}$. Thus, $minA=min\{b{_1},b{_2}...b{_n}\}$.So, we gain the proof.






%Ex2.5
\section{Exercise 2.5}
\subsection{i}


Let 
$z{_1}=(a{_1},b{_1})$,$z{_2}=(a{_2},b{_2})$ $(a, b \in \mathbb{R})$,then $z{_1}+z{_2}=(a{_1}+b{_1},a{_2}+b{_2})$.\par
For the first inequation, we can get the modulus :
$$|z{_1}+z{_2}|=\sqrt{(a{_1}+a{_2})^2+(b{_1}+b{_2})^2}$$
 $$ and  \ \ |z{_1}|+|z{_2}|=\sqrt{(a{_1}+a{_2})^2}+\sqrt{(b{_1}+b{_2})^2}$$
Thus:
$$|z{_1}+z{_2}|^2=(a{_1}+a{_2})^2+(b{_1}+b{_2})^2$$
 $$ and  \ \ (|z{_1}|+|z{_2}|)^2=(a{_1}+a{_2})^2+(b{_1}+b{_2})^2+2\sqrt{(a{_1}+a{_2})^2+(b{_1}+b{_2})^2}$$
because  $\displaystyle2\sqrt{(a{_1}+a{_2})^2+(b{_1}+b{_2})^2}>0$,
 thus we get the proof:
$$|z{_1}+z{_2}|\le|z{_1}|+|z{_2}|$$
\par\par
For the latter equation:
$z{_1}z{_2}=(a{_1}a{_2}-b{_1}b{_2},a{_1}b{_2}+a{_2}b{_1})$  \par so its modulus is$$\displaystyle|z{_1}z{_1}|=(a{_1}a{_2}-b{_1}b{_2},a{_1}b{_2}+a{_2}b{_1})$$
$$and \ \ |z{_1}||z{_2}|=\displaystyle\sqrt{a{_1}^2+a{_2}^2}\sqrt{b{_1}^2+b{_2}^2}$$
Thus:
\begin{align*}
|z{_1}z{_2}|^2&=(a{_1}a{_2}-b{_1}b{_2})^2+(a{_1}b{_2}+a{_2}b{_1})^2
\\&=(a{_1}^2a{_2}^2+b{_1}^2b{_2}^2-2a{_1}a{_2}b{_1}b{_2})+(a{_1}^2b{_2}^2+a{_2}^2b{_1}^2+2a{_1}a{_2}b{_1}b{_2})
\\&=a{_1}^2a{_2}^2+b{_1}^2b{_2}^2+a{_1}^2b{_2}^2+a{_2}^2b{_1}^2
\end{align*}
\begin{align*}
and \ \ (|z{_1}||z{_2}|)^2&=(a{_1}^2+b{_1}^2)(a{_2}^2+b{_2}^2)\\
&=a{_1}^2a{_2}^2+b{_1}^2b{_2}^2+a{_1}^2b{_2}^2+a{_2}^2b{_1}^2
\end{align*}
So we get the proof,$$|z{_1}z{_2}|=|z{_1}||z{_2}|$$



\subsection{ii}

$\quad z=(a, b)\, (a, b \in \mathbb{R}) \quad $ So, $z+2=(a+2, b) \quad z-1=(a-1, b) \\$
Then, we can get:
\begin{align*}
|z+2|={(a+2)^{2}+b^{2}} \\
|z-1|={(a-1)^{2}+b^{2}} \\
\end{align*}
And
\begin{align*}
|z+2|^{2}=(a+2)^{2}+b^{2} \\
|z-1|^{2}=(a-1)^{2}+b^{2}
\end{align*}
Then ,we let $$|z+2|^{2}=(a+2)^{2}+b^{2}\le|z-1|^{2}=(a-1)^{2}+b^{2}$$
And simplify the inequation:
$$6a\le3$$ $$  a\le -\frac{1}{2}$$
So we can get :for complex number $z \in \mathbb{C}$ ,there must exist$\displaystyle Re(z)\le-\frac{1}{2}$ to satisfy the inequation.









\subsection{iii}
Let 
$z{_1}=(a{_1},b{_1})$,$z{_2}=(a{_2},b{_2})$ $(a, b \in \mathbb{R})$,then $z{_1}+z{_2}=(a{_1}+b{_1},a{_2}+b{_2})$.\par Thus ,we can get :
$z_{1}+z_{2}=(a_{1}+a_{2}, b_{1}+b_{2}) \,\, ,
z_{1}-z_{2}=(a_{1}-a_{2}, b_{1}-b_{2}) $
\begin{align*}
|z_{1}+z_{2}|^{2}+|z_{1}-z_{2}|^{2}&=(a_{1}+a_{2})^{2}+(b_{1}+b_{2})^{2}+(a_{1}-a_{2})^{2}+(b_{1}-b_{2})^{2}\\
&=2(a_{1}^{2}+a_{2}^{2})+2(b_{1}^{2}+b_{2}^{2})
\end{align*}
And because
$|z_{1}|=a_{1}^{2}+b_{1}^{2} \,\, ,
|z_{2}|^{2}=a_{2}^{2}+b_{2}^{2} $ \par Thus,
$$2(|z_{1}|^{2}+|z_{2}|^{2})=2(a_{1}^{2}+a_{2}^{2})+2(b_{1}^{2}+b_{2}^{2})$$
Then we get the proof:
$$|z_{1}+z_{2}|^{2}+|z_{1}-z_{2}|^{2}=2(|z_{1}|^{2}+|z_{2}|^{2})$$





\end{document}


